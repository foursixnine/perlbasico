\documentclass[spanish]{article}
\usepackage[T1]{fontenc}
\usepackage[utf8]{inputenc}
\usepackage{babel}
\author{Cooperativa Venezolana de Tecnologías Libres R.S.}
\title{Curso Perl Nivel Básico}
\begin{document}
\maketitle 

\section{Curso de Perl Nivel Básico\label{Curso_de_Perl_Nivel_B_sico}\index{Curso de Perl Nivel Básico}}


Este curso, se esta escribiendo usando Pod::PseudoPod, para poder verlo va a
necesitar instalar los siguientes módulos vía CPAN:

\begin{description}

\item[{File::Path}] \mbox{}
\item[{File::Spec::Functions}] \mbox{}
\item[{Pod::PseudoPod}] \mbox{}
\item[{Pod::PseudoPod::HTML}] \mbox{}
\item[{App::pod2pdf}] \mbox{}\end{description}


Una vez instalado los módulos, solo siga las instrucciones de la siguiente
sección para poder generar los capítulos en el formato que prefiera, HTML o 
PDF.

\subsection*{Instrucciones para ver el curso\label{Instrucciones_para_ver_el_curso}\index{Instrucciones para ver el curso}}


El primer paso es generar los capítulos, esto se debe hacer desde la carpeta
raíz del curso:

\begin{verbatim}
    $ perl build/tools/build_chapters.pl
\end{verbatim}


El segundo paso, es transformar los archivos POD al formato que prefiera. Si
prefiere HTML entonces debe hacer lo siguiente

\begin{verbatim}
    $ perl build/tools/build_html.pl
\end{verbatim}
\subsection*{Instrucciones para escribir capítulos\label{Instrucciones_para_escribir_cap_tulos}\index{Instrucciones para escribir capítulos}}


Los capítulos se encuentran en la carpeta \textbf{sections/}, y siguen la
nomenclatura chapter\_XX.pod, donde XX es el número del capítulo, comenzando por
00, 01 y así sucesivamente.



Los capítulos se escriben usando el módulo Pod::PseudoPod, que es una extensión
de POD para crear libros. En este sentido recomendamos la lectura de los
siguientes dos manuales.

\begin{verbatim}
    $ perldoc perlpod
\end{verbatim}
\begin{verbatim}
    $ perldoc Pod::PseudoPod::Tutorial
\end{verbatim}


Tambien puede fijarse en la sintaxis utilizada en los capítulos ya existentes.



Adicionalmente, se recomienda evitar el uso de tabulador dentro de los ficheros
POD, configure su editor de texto para que reemplace los TABs por 4 espacios en
blanco.

\subsection*{Instrucciones para escribir ejercicios\label{Instrucciones_para_escribir_ejercicios}\index{Instrucciones para escribir ejercicios}}


Los ejercicios de cada capítulo se encuentran en un archivo llamado
ejercicios00.pod en donde 00 corresponde al número del capítulo dentro de la
carpeta \textbf{sections/}.



Las recomendaciones en la sección anterior aplican para el desarrollo de los
ejercicios también.



Cada archivo de ejercicios00.pod deberia tener un archivo correspondiente a los
ejercicios resueltos llamado respuestas00.pod, de igual manera 00 
corresponde al número del capítulo.


\end{document}
